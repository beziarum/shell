\documentclass[12pt]{article}

\usepackage[T1]{fontenc} 
\usepackage[utf8]{inputenc}
\usepackage[francais]{babel}
\usepackage{graphicx}
\usepackage{hyperref}
\usepackage{url}
\usepackage{color}
\usepackage[usenames,dvipsnames]{xcolor}
\usepackage{alltt}
\usepackage{tocbibind}


\title{Rapport Projet Programmation Système}
\author{Beziau Paul, Bazalgette Martin, Borde Antoine}

\begin{document}
\maketitle
\tableofcontents

\newpage
\section{Présentation}

L'objectif de ce projet étais d'implémenter un shell capable d'interpréter les expression de base
utilisable avec bash et d'éxétcuter des procéssus. En réalité une bonne partie du code du shell
nous été deja fournit. Tout l'aspect de l'analyse syntaxique été déja implémenté dans les fichier
donné dans le sujet.\newline

On peut diviser le travail à effectuer en 4 partie qui sont:
\begin{itemize}
 \item Implémenter la possibilité pour notre shell d'éxécuter des
 commande externe ou des progrmme en tout genre n'appartenant pas au shell
 \item Donner au shell la capacité d'interpréter les expression courante
 utilisable sous bash, comme les redirections ou les pipes
 \item Ecrire un certain nombre de commande interne (des fonctions appartennant
 au shell en lui meme) tout en faisant attention à les rendre utilisable avec
 les expressions précédement cité
 \item L'implémentation d'une commande interne "remote" qui nous permetra d'éxécuter
 et controler des shells distants en utilisant ssh.\newline 
\end{itemize}

Nous verrons au travers de ce rapport comment nous avons choisit d'implémenter ces différentes
fonctionnalité.

\newpage
\section{Commande externe}


\newpage
\section{Expression}


\newpage
\section{Commande interne}


\newpage
\section{Remote shell}


\newpage
\section{Conclusion}



\end{document}
