\documentclass[12pt]{article}

\usepackage[T1]{fontenc} 
\usepackage[utf8]{inputenc}
\usepackage[francais]{babel}
\usepackage{hyperref}
\usepackage{url}
\usepackage{color}
\usepackage[usenames,dvipsnames]{xcolor}
\usepackage{verbatim}
\usepackage{tocbibind}


\title{Rapport Projet Programmation Système}
\author{Beziau Paul, Bazalgette Martin, Borde Antoine}

\begin{document}
\maketitle
\tableofcontents

\newpage
\section{Présentation}

L'objectif de ce projet était d'implémenter un Shell capable d'interpréter les expressions de base
utilisable avec bash et d'exécuter des processus. En réalité, une bonne partie du code du Shell
nous était déjà fourni. Tout l'aspect de l'analyse syntaxique était déjà implémenté dans les fichiers
donnés dans le sujet.\newline

On peut diviser le travail à effectuer en 4 parties qui sont:
\begin{itemize}
 \item Implémenter la possibilité pour notre Shell d'exécuter des
 commandes externes ou des programmes en tout genre n'appartenant pas au Shell
 \item Donner au Shell la capacité d'interpréter les expressions courantes
 utilisables sous bash, comme les redirections ou les pipes
 \item Écrire un certain nombre de commandes interne (des fonctions appartenant
 au Shell en lui-même) tout en faisant attention à les rendre utilisables avec
 les expressions précédemment citées.
 \item L'implémentation d'une commande interne "remote" qui nous permettra d'exécuter
 et contrôler des shells distants en utilisant ssh.\newline 
\end{itemize}

Nous verrons au travers de ce rapport comment nous avons choisi d'implémenter ces différentes
fonctionnalités.


\newpage

\section{Commande externe}
 
 Le début de ce projet avait déjà été bien entamé en tp. Nous nous sommes donc
 inspirés de ce qui avait été fait en classe. Nous avons créé une structure "contexte"
 qui contiendra à terme tous les éléments dont nous aurons besoin. \ref{Eval.h}
 
 Ensuite, nous avons implémenté une deuxième structure "assoc" qui possède un champ "expr\_t"
 et un pointeur de fonction reliant le type d'expression a la fonction de traitement
 qui lui est associé.
 
 Pour finir, nous avons créé la fonction "get\_expr" qui retourne la fonction associée au 
 type de l'expression et l'appel en lui passant les paramètres de l'expression et son 
 contexte.\ref{struct et fonctionnement}
 
 Ce travail préparatoire permettra de rendre beaucoup plus aisée la suite du développement
 du projet
 
 En effet ne nous restait alors plus qu'à implémenter une fonction pour intégrer le comportement
 de chaque expression.
 
 \subsection{Commande simple en avant-plan}
 
 L'exécution de commande simple en avant plan est implémentée avec un simple fork().
 On récupère les arguments de la commande rentrée et l'on fait un "execvp" de cette commande
 avec ces paramètres du côté de fils. Pendant ce temps le père attend la fin de l'exécution
 et récupère la valeur de retour de son fils.\ref{expr_simple}
 
 \subsection{Commande simple en arrière plan}
 
 Pour traiter des commandes en arrière-plan, nous avons rajouté un champ "background"
à notre structure contexte qui passe à vraie si le caractère \& est interprété.
 Ensuite, il suffit de dire que dans ce contexte on lance le fork() mais on n’attend pas
la fin de l'exécution du fils.\ref{cmd simple arriere plan}
 
 
 \subsection{Élimination des zombies}

 Pour éliminer les zombies causés par les commandes en arrière plan nous avons choisi d'effectuer
 à chaque tours de boucles une série de wait non bloquant, cette série se terminant dès qu'un des
 wait renvoie 0. \ref{main}
 

\newpage
\section{Expression}

Dans cette partie, le travail consiste à implémenter l'interprétation d'expression courante
normalement utilisable avec bash tel que les redirections ">" ou les pipes "|".
En réalité, ces expressions sont déjà reconnues par le Shell. Il ne nous reste plus qu'à utiliser
nos structures pour relier chaque expression à son comportement spécifique et d'implémenter le
dit comportement.

Comme expliquer plus haut le gros travail préparatoire que nous avons effectué nous permettra
de traiter avec une relative simplicité cette partie. En effet, il nous suffira la plupart du temps 
d'exploiter l'arborescence déjà implémentée dans le sujet, ainsi que nos structures, pour concevoir
les différents comportements.\ref{autre expression}

\subsection{Les séquences}

Une expression séquentielle consiste à exécuter la commande de gauche puis celle de droite.
Il nous suffit donc d'appeler dans l'ordre avec notre fonction "get\_expr" le fils gauche puis le fils
droit de notre expression ";".\ref{autre expression}

\subsection{les expressions booleene}

Nous avons suivi le même principe pour les expressions booléennes. On appelle le fils gauche puis
le fils droit de l'expression.
À la différence que le fils droit n'est appelé que si la commande gauche à bien était exécuté dans le
cas du "et".Et le fils droit n'est appelé qu’en cas d'échec du fils gauche dans le cas du "ou".\ref{et et ou}

\subsection{Les sous-expressions}

Il s'agit sans doute de l'expression la plus simple à gérer. Il suffit d'appeler le fils
gauche de l'expression. La gestion de l'arborescence fera le reste.\ref{sousshell}

\subsection{Les redirections}

Pour gérer les redirections, nous avons rajouté dans notre structure contexte des descripteurs
de fichier correspondant aux entrées et sortie standard des processus. Il nous suffit ensuite 
d'ouvrir le fichier passé en paramètre sur les bons descripteurs et d'appeler la commande.
Les entrées et sorties seront ensuite remises à zéro par notre fonction "swapfd". \ref{redirection}

\subsection{Les pipes}

Un pipe est composé de deux commmande. La première sera exécuté en arrière plan et on changera
sa sortie standard par l'entrée d'un pipe. Pour la seconde on a juste à remplacer son entrée par
la sortie du pipe précédant. Ces deux parties ont été exécuté simplement en utilisant les
redirections et l'arrière plan précédement codé. En plus de celà il fallait fermer les deux
côtés du pipe dans le père, pour celà nous avons rajouté un champs dans la structure contexte.\ref{pipe}


\subsection{Les expressions récursives}

Dès le début de la conception la structure de notre programme nous permettait d'abstraire l'ordre
des expressions, et donc les séquences de commandes. Cette partie ne nous a donc pausé aucun problème :
nous ne nous en somme jamais occupé, cette partie marchant dès le début.


\newpage
\section{Commande interne}

Dans cette partie, nous avons dû implémenter directement le comportement de quelques
commandes dans notre Shell. Ces commandes n'étaient pas très difficiles à implémenter. La plupart
du temps, il suffisait de trouver le bon appel système à utiliser. À noter tout de même
que nous n'avons implémenté que les fonctionnalités de base offertes par ces commandes, nous
ne nous sommes pas amusés à ajouter la gestion de tous les paramètres, ce qui aurait 
représenté un travail très fastidieux.\ref{cmd_de_base}

\subsection{Fonction}
\subsubsection{echo}
 La commande "echo" n'est rien d'autre qu'une fonction qui écrit les paramètres qu'elle reçoit
 sur la sortie standard. Un simple printf suffit pour cette commande. \ref{echo}
 
\subsubsection{date}
 Pour cette commande, nous avons dû faire quelques recherches sur les fonctions fournies par la librairie "time".
 Nous avons décidé de créer une structure "tm" et d'utiliser la fonction "strftime" qui
 sert justement à afficher l'heure selon son propre format qu'on peut définir à la volée. \ref{date}
 
 \subsubsection{cd}
 La commande "cd" a requis d'utiliser l'appel système "chdir" qui ouvre le répertoire passé en paramètre
 Nous avons veillé à récupérer la variable d'environnement "HOME" avec getenv pour les cas ou "cd" est
 appelé sans paramètre. Nous avons également tenté de prévenir d'éventuelles erreurs(comme les
 répertoires non existants). \ref{cd}
 
 \subsubsection{pwd}
 Cette commande appelle la fonction getcwd qui copie le chemin absolu du répertoire 
 courant dans un buffer.
 Le buffer est initialisé avec la taille PATH\_MAX, qui est une macro du fichier limites.h,
 et qui correspond à la taille maximale d'un chemin. Il suffit ensuite d'afficher le buffer. \ref{pwd et hostname}
 
 \subsubsection{hostname}
 Cette fonction a été implémentée d'une manière semblable à la fonction PWD.
Tout d'abord, nous avons du récupérer le nom de l'hôte local grâce à l'appel système gethostname, qui copie ce nom dans un buffer. 
La taille du buffer est initialisée à l'aide de la macro HOST\_NAME\_MAX qui correspond à la taille maximale d'un nom de localhost. 
On affiche ensuite ce dernier. \ref{pwd et hostname}
 
 \subsubsection{exit}
 La commande exit permet de quitter le Shell. 
 Elle peut être exécutée avec ou sans paramètres. 
 S'il n'y a pas de paramètre, on ferme le Shell en renvoyant la valeur 0. 
 Sinon, on ferme le Shell en renvoyant la valeur entière du premier paramètre. Il est donc inutile de donner plusieurs paramètres à cette commande. \ref{exit et kill}
 
 \subsubsection{kill}
 Pour la commande kill nous avons simplement utilisé l'appel système kill. 
 Notre commande peut gérer le paramètre de choix de signal à envoyer. Il peut l'envoyer à tous les processus
 rentrés en paramètres.
 Pour gérer l'erreur en cas d'absence de paramètres nous avons touché directement à la valeur
 de errno la passant à EINVAL (invalid argument). \ref{exit et kill}
 
 \subsubsection{history}
 La commande history renvoie l'historique des commandes entrées par l'utilisateur. 
 Elle nécessite la gestion d'un historique dans le Shell. 
 Pour cela il a fallu inclure le fichier history.h permettant d'utiliser un historique. 
 Pour ce faire, on appelle dans un premier temps la fonction using\_history dans le main avant la boucle principale.
 Ainsi on pourra utiliser les fonctions de gestions d'historique. 
 À chaque fois qu'une fonction est exécutée, elle est ajoutée à l'historique grâce à la fonction add\_history. 
 Ensuite, notre fonction history récupère la liste des fonctions appelées, et les affiches une par une. 
 On peut également exécuter la commande avec un argument n, pour afficher uniquement les n dernières commandes. \ref{history}
 
 \subsection{Intégration au Shell}
 
 La difficulté de cette partie n'était pas tant l'implémentation des fonctions en elles mêmes,
 mais surtout le fait de rendre compatible tout ce qui avait été fait avant avec ces commandes.
 En effet, en temps normal, les commandes internes n'utilisent pas de fork et ne permettent pas
 de rendre la main au shell. Dans un premier temps nous avons donc envisagé d'utiliser des threads
 pour exécuter les deux parties du programme en même temps. Cepandant les threads d'un même processus
 se partageant le même tableau de descripteur de fichier, cette stratégie n'était pas compatible
 avec la redirection de flux. Finalement nous avons décidé d'exécuter les commandes internes dans
 un processus fils si et seulement si elles devaient l'être en arrière plan.
 Afin de ne pas nous encombrer d'une forêt de if ou d'un « switch/case » conséquent, nous avons
 décidé de réutiliser un tableau d'association, mettant en relation le nom d'une commande interne
 avec la fonction traitant cette commande.  \ref{expr_simple}
 

\newpage
\section{Remote Shell}

La commande interne «~remote~» doit permetre d'exécuter simplement des commandes sur une machine
distante ou sur un ensemble de machines distantes en utilisant ssh en interne. Dans un premier 
temps nous avons essayé d'ouvrir la connection dès le «~remote add~», comme suggéré dans le 
sujet, mais nous nous sommes très rappidement confronté à plusieurs problème parmis lesquels
principalement la difficulté pour savoir quand une commande distante était terminé, et donc quand
il falait rendre la main au shell. Nous avons alors décidé de n'établire la connection qu'au moment
d'exécuter une des commandes distantes, et de la fermer après la fin de cette dernière.\newline

Une machine n'est donc représenté que par son nom, Bien que nous ayons gardé en pratique une
structure pour la représenter, n'étant pas sûr de la façon dont notre code allait évoluer et
donc si nous n'allions pas avoir besoin de sauvegarder plus d'informations que simplement le
nom de la machine distante.


\subsection{Shell faussement distant}

Nous avons tout d'abord créé une commande «~remote localhost~» qui ne lançait pas la commande
suivante via ssh mais en local, en utilisant cepandant notre shell en tant qu'interface entre le shell
controlé par l'utilisateur et le programme demandé. Pour (re)lancer notre shell nous construisons une
expression demandant à lancer notre shell avec l'option -r qui active le mode distant. La commande
“distante” est lancée en ayant écrit dans un pipe branché sur l'entrée du nouveau shell.

\subsection{Gestion des listes (add, remove et list)}

Les deux premières commandes ont d'abord été pensé pour gérer respectivement l'initialialisation
et la fermeture du shell distant, ce qui complexifiait particulièrement leur structure. Par la
suite, ayant décidé de n'effectuer ces actions qu'au moment d'exécuter une commande distante, nous
avons pu les simplifiers.

\subsection{Exécution distante}

Pour cette partie nous avons en très grande partie réutilisé le code créé pour le shell faussement
distant. En effet nous avons repris l'idée de créé une expression valide et de l'exécuter à l'aide
de la première partie du programme. simplement au lieu de lancer notre shell nous lançons ssh en
précisant dans les paramètre le programme à appeler, c'est à dire notre propre shell, et nous appelons
le programme demandé en écrivant via un pipe dans l'entrée standard de ssh, donc de notre shell distant.

\subsection{xcat}

Cette dernière partie a finalement été facilement traitée : il nous a suffit de piper la sortie de ssh
sur l'entrée du programme xcat.sh, en donnant à ce dernier les paramètres nécessaires pour qu'il ne se coupe pas
sitôt ssh finit et qu'il affiche en titre le nom de la machine sur laquelle la commande a été exécutée

\ref{remote shell}





\newpage
\section{Conclusion}

Notre plus grande réussite dans ce projet est très clairement le travail effectué au
début du projet. En effet, nous avons su tout de suite préparer un terrain propice au
développement de notre shell. Notamment grâce à l'implémentation de notre système de structure et 
de fonction définissant le comportement des expressions\ref{struct et fonctionnement}.
La preuve en est qu'une fois arrivé à l'étape de la gestion de la récursivité des expressions, 
cette option était en fait déjà fonctionnel depuis le début.\newline

Si les trois premières parties du développement ne nous ont pas trop posé de problème, ce n'est pas le cas de la dernière visant à contrôler les shell distants dont nous sommes moins satisfaits.Outre les difficultés que nous avons rencontrées, nous ne sommes pas certains que le résultat obtenu soit clairement celui attendu. Ce qui est particulièrement frustrant dans la mesure ou cette partie nous a demandé plus de travail que les trois autres réunies.\newline

Ce projet nous a permis de renforcer nos notions en programmation-système. Ainsi que de mieux cerner le fonctionnement d'un shell. Nous en avons également profité pour renouer avec latex, pendant la rédaction du rapport, dont les performances nous satisfont grandement.


\newpage
\section{Annexes}

Ci-joint en annexes le code source du projet. Nos fichiers sont disponible sur le dépot git suivant:
 \href{https://github.com/beziarum/shell.git}{ici}.

 \scriptsize
\subsection{Shell.c}
\label{main}
\begin{verbatim}

 int main (int argc, char **argv)
{
    for(int i=1; i<argc; i++)
	if(strcmp("-r",argv[i])==0)
	    interactive_mode=false;
    using_history();
  while (1){
    if (my_yyparse () == 0) {  /* L'analyse a abouti */
      fflush(stdout);
      status = evaluer_expr(ExpressionAnalysee);
      expression_free(ExpressionAnalysee);


      //on exécute un waitpid non bloquant (WNOHANG)
      //tant que sa valeur de retour vaut autre chose que 0 alors
      //c'est le pid d'un processus qui c'est terminé
      int pid,status;
      while( (pid = waitpid(-1,&status,WNOHANG)) > 0)
	  printf("le processus %d est fini (%d)\n",
		 pid,
		 WIFEXITED(status) ? WEXITSTATUS(status) : 128+WTERMSIG(status));
    }
    else {
    }
  }
  return 0;
}
\end{verbatim}


\subsection{Évaluation.c}

\subsubsection{Structure et fonctionnement}
\label{struct et fonctionnement}
\begin{verbatim}
 typedef struct assoc
{
  expr_t expr;
  int (*data) (Expression *, Contexte *);
} assoc;

pid_t lpid = 0;

/* Tableau permettant d'associer un type d'expréssion avec la fonction qui la traite */
assoc tab_expr[] = { {SIMPLE, expr_simple},
		     {BG, expr_bg},
		     {SEQUENCE, expr_sequence},
		     {SEQUENCE_ET, expr_sequence_et},
		     {SEQUENCE_OU, expr_sequence_ou},
		     {REDIRECTION_I, expr_redirection_in},
		     {REDIRECTION_O, expr_redirection_o},
		     {REDIRECTION_A, expr_redirection_a},
		     {SOUS_SHELL, expr_sous_shell},
		     {VIDE, expr_vide},
		     {REDIRECTION_E, expr_redirection_er},
		     {PIPE, expr_pipe}
};

int (*get_expr (expr_t expr)) (Expression *, Contexte *)
{
  int taille_tab_expr = sizeof (tab_expr) / sizeof (assoc);
  for (int i = 0; i < taille_tab_expr; i++)
    if (tab_expr[i].expr == expr)
      return tab_expr[i].data;
  return expr_not_implemented;
}

/*
 * Initialise le contexte passé en parametre aux valeurs par défaut
 */
void
initialiser_contexte (Contexte * c)
{
  c->bg = false;
  c->fdin = STDIN_FILENO;
  c->fdout = STDOUT_FILENO;
  c->fderr = STDERR_FILENO;
  c->fdclose = -1;
  c->tube = NULL;
}

/*
 * copie les valeurs de c1 dans c2
 */
void
copier_contexte (Contexte * c1, Contexte * c2)
{
  c2->bg = c1->bg;
  c2->fdin = c1->fdin;
  c2->fdout = c1->fdout;
  c2->fderr = c1->fderr;
  c2->fdclose = c1->fdclose;
  c2->tube = c1->tube;
}

/*
 * applique un contexte et sauvegarde les anciens fd si save
 * vaut vrai
 */
void
appliquer_contexte (Contexte * c, bool save)
{
  swapfd (&(c->fdin), STDIN_FILENO, save);
  swapfd (&(c->fdout), STDOUT_FILENO, save);
  swapfd (&(c->fderr), STDERR_FILENO, save);
  if (c->fdclose != -1)
    close (c->fdclose);
}

/*
 * effectue un dup2 sur la valeur pointée par fdnew  et sur fdold
 * si save vaut vrai alors fdold est sauvegardé pour pouvoir effectuer 
 * l'opération inverse.
 */
void
swapfd (int *fdnew, int fdorigin, bool save)
{
  if (*fdnew != fdorigin)
    {
      int tmp;
      if (save)
	tmp = dup (fdorigin);
      dup2 (*fdnew, fdorigin);
      if (save)
	*fdnew = tmp;
      else
	close (*fdnew);
    }
}

int
evaluer_expr (Expression * e)
{
  Contexte *c = malloc (sizeof (Contexte));
  initialiser_contexte (c);
  return get_expr (e->type) (e, c);
}

/*
 * renvoi le pid du dernier fils créé
 */
pid_t
get_last_pid ()
{
  return lpid;
}
\end{verbatim}

\subsubsection{Expression simple}
\label{expr_simple}
\begin{verbatim}
int
expr_simple (Expression * e, Contexte * c)
{
  int (*intern) (char **) = get_intern (e->arguments[0]);
  if (intern != NULL && c->bg != true)	// si la commande est une commande interne
    // et en avant-plan
    {
      appliquer_contexte (c, true);	// on applique le contexte
      int ret = intern (e->arguments);	// on exécute la commande interne
      appliquer_contexte (c, false);	// et on restaure le contexte
      return ret;

    }
  else
    {				//pour le fils…
      pid_t pid;
      pid = fork ();
      if (pid == 0)
	{
	  appliquer_contexte (c, false);	// on applique le contexte
	  if (intern != NULL)	// si c'est une commande interne
	    {
	      exit (intern (e->arguments));	// on l'éxécute, puis on coupe
	    }
	  else
	    {
	      execvp (e->arguments[0], e->arguments);	// sinon c'est une commande externe,
	      // donc on lui passe la main
	      perror (e->arguments[0]);
	      exit (1);
	    }
	}
      else
	{			// pour le père
	  lpid = pid;		// on met à jour le pid du dernier fils créé
	  if (c->tube != NULL)	// s'il existe on ferme le tube
	    {
	      close (c->tube[0]);
	      close (c->tube[1]);
	    }
	  if (c->bg)		// si le fils est exécuté en arrière plan alors on
	    return 0;		// redonne la main au shell
	  int status;
	  waitpid (pid, &status, 0);	// sinon on attend qu'il termine et on renvoie son code de retour
	  return WIFEXITED (status) ? WEXITSTATUS (status) : 128 +
	    WTERMSIG (status);
	}
    }
}
 
\end{verbatim}

\subsubsection{Autre Expressions}
\label{autre expression}
\begin{verbatim}
 int
expr_not_implemented (Expression * e, Contexte * c)
{
  fprintf (stderr, "fonctionnalité non implémentée\\n");

  return 1;
}

/* 
 * Fonction traitant l'expression vide. Rien ne se passe. 
 */
int
expr_vide (Expression * e, Contexte * c)
{
  return EXIT_SUCCESS;
}

/*
 * Exécute l'expression située à gauche puis celle à droite
 */
int
expr_sequence (Expression * e, Contexte * c)
{
  get_expr (e->gauche->type) (e->gauche, c);
  return get_expr (e->droite->type) (e->droite, c);
}
\end{verbatim}
\label{cmd simple arriere plan}
\begin{verbatim}
/*
 * Fonction traitant les expressions en arrière plan.
 */
int
expr_bg (Expression * e, Contexte * c)
{
  c->bg = true;
  return get_expr (e->gauche->type) (e->gauche, c);
}
\end{verbatim}
\label{pipe}
\begin{verbatim}
int
expr_pipe (Expression * e, Contexte * c)
{
  Contexte *c2 = malloc (sizeof (Contexte));	// on créé une copie du contexte
  copier_contexte (c, c2);

  int *tube = c2->tube = malloc (2 * sizeof (int));	// on garde une copie du tube
  pipe (tube);			// dans le contexte de droite pour
                                // pouvoir le fermer dans le shell

  c->bg = true;			// la commande à gauche sera en arrière plan
  c->fdout = tube[1];		// sa sortie sera redirigée vers l'entrée du pipe
  c->fdclose = tube[0];		// et on fermera la sortie du tube

  c2->fdin = tube[0];		// on redirigera la sortie du tube vers l'entrée de la
                                // seconde commmande
  c2->fdclose = tube[1];	// et on fermera l'entrée du tube

  get_expr (e->gauche->type) (e->gauche, c);	// enfin on exécute les
  return get_expr (e->droite->type) (e->droite, c2);	// deux commandes
}

\end{verbatim}
\label{et et ou}
\begin{verbatim}
/*
 * Traite l'expression SEQUENCE ET
 * Exécute la première commande puis la seconde si la première a
 * renvoyé un succès
 */
int
expr_sequence_et (Expression * e, Contexte * c)
{
  int ret = get_expr (e->gauche->type) (e->gauche, c);
  if (ret != 0)
    return ret;
  else
    return get_expr (e->droite->type) (e->droite, c);
}

/*
 * Traite l'expression SEQUENCE OU
 * Exécute la première commande puis la seconde si la première a
 * renvoyé un echec
 */
int
expr_sequence_ou (Expression * e, Contexte * c)
{
  int ret = get_expr (e->gauche->type) (e->gauche, c);
  if (ret == 0)
    return ret;
  else
    return get_expr (e->droite->type) (e->droite, c);
}
\end{verbatim}
\label{sousshell}
\begin{verbatim}
/*
 * Traite l'expression SOUS_SHELL
 * En pratique on se contante de traiter l'expression de gauche
 */
int
expr_sous_shell (Expression * e, Contexte * c)
{
  return get_expr (e->gauche->type) (e->gauche, c);
}
\end{verbatim}
\label{redirection}
\begin{verbatim}
/*
 * Traite l'expression REDIRECTION_I
 * on indique dans le contexte qu'il faudra utiliser à la place de
 * l'entrée standard le descripteur correspondant au fichier donné
 * en paramètre
 */
int
expr_redirection_in (Expression * e, Contexte * c)
{
  c->fdin = open (e->arguments[0], O_RDONLY);
  return get_expr (e->gauche->type) (e->gauche, c);
}

/*
 * idem pour la sortie standard
 */
int
expr_redirection_o (Expression * e, Contexte * c)
{
  c->fdout = open (e->arguments[0], O_WRONLY | O_CREAT | O_TRUNC, 0660);
  return get_expr (e->gauche->type) (e->gauche, c);
}

/*
 * idem mais le fichier est ouvert en mode append
 */
int
expr_redirection_a (Expression * e, Contexte * c)
{
  c->fdout = open (e->arguments[0], O_WRONLY | O_CREAT | O_APPEND, 0660);
  return get_expr (e->gauche->type) (e->gauche, c);
}

/*
 * idem mais pour la sortie d'erreur
 */
int
expr_redirection_er (Expression * e, Contexte * c)
{
  c->fderr = open (e->arguments[0], O_WRONLY | O_CREAT | O_TRUNC, 0660);
  return get_expr (e->gauche->type) (e->gauche, c);
}
\end{verbatim}

\subsection{Évaluation.h}
\label{Eval.h}
\begin{verbatim}
 /*
 *si Contient le contexte demandé pour la commande à lancer
 */
typedef struct Contexte {
    bool bg; //si elle doit etre en arrière plan
    int fdin;//si l'entrée standart de la commande doit être changé
    int fdout;//idem pour la sortie standard
    int fderr;//idem pour la sortie d'erreur
    int fdclose;//file descriptor qu'il faudra fermer avant de lancer la commande, -1 si aucun
    int* tube;//contient un tube à fermer par le père si besoin
} Contexte;
\end{verbatim}

\subsection{Commandes-internes.c}

\subsubsection{Les commandes internes de bases}
\label{cmd_de_base}
\begin{verbatim}
 void verifier(int b,char* m)
{
  if(!b)
    perror(m);
}

/*
 * renvoi un pointeur vers la fonction traitant la commande
 * interne passée en paramètre si elle existe, NULL sinon
 */
int (*get_intern (char* name)) (char**)
{
  int taille_tab_cmd_intern = sizeof (tab_cmd_intern)/sizeof(assoc);
  for (int i=0; i<taille_tab_cmd_intern; i++)
    if (strcmp(name,tab_cmd_intern[i].name)==0)
      return tab_cmd_intern[i].data;
  return NULL;
}
\end{verbatim}
\label{echo}
\begin{verbatim}
/*
 * Commande permettant d'afficher des chaines de caractères
 */
int echo(char ** arg)
{
  
  int c=1;
  while (arg[c]!=NULL)
    {
      if (arg[c][0]=='$'){
	char *tmp=arg[c]+1;
	printf("%s ",getenv(tmp));
	c++;
      }
      else{
	printf("%s ",arg[c]);    // on affiche simplement ces paramètres sur la sortie standard
	c++;
      }
    }
  printf("\n");
  return EXIT_SUCCESS;
}
\end{verbatim}
\label{date}
\begin{verbatim}
/*
 * Commande affichant la date
 */
int date(char ** arg)
{
  char c[256]; 
  time_t tmp = time(NULL);                              // on recupère le temps en secondes
  struct tm * t =localtime(&tmp);                       // on génère une structure tm
  strftime(c, sizeof(c), "%A %d %B %Y, %X (UTC%z)",t);  // on affiche selon le format français
  printf("%s\n", c);
  return EXIT_SUCCESS;
}

\end{verbatim}
\label{cd}
\begin{verbatim}
/*
 * Commande permettant de changer de répertoire courant
 */
int cd (char ** arg)
{
  int r;
  if (arg[1]==NULL)
  {
    r = chdir(getenv("HOME"));       // cas d'un retour au home cd sans paramètre
    verifier(r!=-1,"cd");
  }
  else
  {
    r = chdir(arg[1]);               // cas classique : on apelle juste chdir avec le nom du dossier
    verifier(r!=-1, "cd");
  }
  return r;
}

\end{verbatim}
\label{pwd et hostname}
\begin{verbatim}
/* 
 * Commande qui affiche le répertoire courant
 */
int pwd(char ** arg) 
{
  char pwd[PATH_MAX];
  getcwd(pwd, sizeof(pwd));
  printf("%s\n", pwd);
  return EXIT_SUCCESS;
}


/*
 * Commande qui affiche le nom de l'hote local
 */
int hostname(char ** arg) 
{
  char hostname[HOST_NAME_MAX +1];
  gethostname(hostname,sizeof(hostname));
  printf("Hostname : %s\n", hostname);
  return EXIT_SUCCESS;
}

\end{verbatim}
\label{exit et kill}
\begin{verbatim}
/*
 * Commande permettant que quitter le shell
 */
int my_exit(char ** arg) 
{
  if (arg[1] != NULL)         // si il y a un argument
    exit(atoi(arg[1]));       // on quitte le shell en renvoyant la valeur de l'argument
  else
    exit(0);
}


/*
 * Commande kill
 * on va utiliser la fonction kill de la libc
 */
int killShell (char ** arg)
{
  if(arg[1]==NULL)
  {
    errno=EINVAL;
    perror("kill");
    return EXIT_FAILURE;
  }
  int ret;
  if (arg[1][0]!= '-')
  {
    int c = 1;
    while (arg[c]!=NULL)
    {
      ret = kill (atoi(arg[1]),SIGTERM);        // cas par defaut sans signal passé en paramètre
      verifier(ret!=-1, "kill");
      c++;
    }
  }
  else
  {
    char *sign = arg[1]+1;                      // on récupère le premier paramètre sans le "-"
    int x = atoi (sign);
    int c=2;
    while (arg[c]!=NULL)
    {
      ret = kill(atoi(arg[c]),x);               // et on lance le signal
      verifier(ret!=-1, "kill");
      c++;
    }
  }
  return ret;
}

\end{verbatim}
\label{history}
\begin{verbatim}
/*
 * Commande qui affiche l'historique du shell. On peut l'appeler avec un argument
 * pour obtenir l'historique des n dernières commandes 
 */
int history(char ** arg) 
{
  if (arg[2] != NULL)
  {
    errno=EINVAL;
    perror("history");
    return EXIT_FAILURE;
  }
  HIST_ENTRY ** hystory_list = history_list ();                       // on crée une variable contenant l'historique
  int treshold = history_length;
  if (arg[1] != NULL && atoi(arg[1]) < history_length)                // si il y a un argument, 
    treshold = atoi(arg[1]) +1;                                       // et qu'il est inférieur au nombre d'éléments de l'historique
  for (int i = history_length - treshold; i < history_length; i++)    // on affiche les n derniers rangs de l'historique
    printf ("%d: %s\n", i + history_base, hystory_list[i]->line);     // (sans compter la commande history qu'on vient de lancer)
  return EXIT_SUCCESS;
}
\end{verbatim}

\subsubsection{remote shell}
\label{remote shell}
\begin{verbatim}
 typedef struct remote_machine {   // structure représentant une machine distante
    char* name;
} remote_machine;

/* Tableau contenant des pointeurs vers des structures de machines distantes */
remote_machine **tab_machines=NULL;

/* Taille initiale du tableau */
int tab_length =0;

/* Nombre de machines présentes dans le tableau */
int nb_machine=0;

int remote_localhost(char** param);
int remote_add(char** param);
int remote_remove(char ** param);
int remote_list(char ** param);
int remote_all(char** param);
int remote_cmd_simple(char** param);
			  
int (*get_remote (char* name)) (char**);

/* Tableau associant les fonctionnalités de la commande remote avec les traitants */ 			  
assoc tab_remote[] = {{"localhost", remote_localhost},
		      {"add", remote_add},
		      {"remove", remote_remove},
		      {"all", remote_all},
		      {"list", remote_list}};

/*
 * fonction interne remote, elle se contente d'appeler la fonction s'occupant de la 
 * sous-commande correspondant à la commande actuelle
 */
int remote(char** params)
{
    int (*cmd_remote) (char**)=get_remote(params[1]);
    if(cmd_remote!=NULL)
	return cmd_remote(params);
    else
    {
	fprintf(stderr,"sous commande de remote inconnue (%s)\n",params[1]);
	return EXIT_FAILURE;
    }
}

/*
 * parcourt le tableau d'association, renvoi la fonction s'ocupant de « remote
 * <name> » si elle existe, sinon remote_cmd_simple
 */
int (*get_remote (char* name)) (char**) 
{
  int taille_tab_remote = sizeof (tab_remote)/sizeof(assoc);
  for (int i=0; i<taille_tab_remote; i++)
    if (strcmp(name,tab_remote[i].name)==0)
      return tab_remote[i].data;
  return remote_cmd_simple;
}

/*
 * Commande créant un shell faussement distant sur la machine locale 
 */
int remote_localhost(char** param)
{
    char** param_remote=malloc(3*sizeof(char**));//on créé la liste de la commande à exécuter ("./Shell -r")
    param_remote[0]=strdup("./Shell");
    param_remote[1]=strdup("-r");
    param_remote[2]=NULL;
    
    Expression* e=ConstruireNoeud(SIMPLE,NULL,NULL,param_remote);//on construit l'expression
    e=ConstruireNoeud(BG,e,NULL,NULL);
    Contexte* c=malloc(sizeof(Contexte));
    
    initialiser_contexte(c);
    int tube[2];
    pipe(tube);
    c->fdin=tube[0];
    c->fdclose=tube[1];
    param+=2;
    while(*param!=NULL)//on écrit les paramètres de la fonction sur l'entrée du nouveau shell
    {
	write(tube[1],*param,strlen(*param));
	write(tube[1]," ",1);
	param++;
    }
    write(tube[1],"\n",1);
    get_expr(BG)(e,c);//et on l'exécute
    close(tube[1]);
    expression_free(e);
    free(c);
    int status;
    waitpid(get_last_pid(),&status,0);
    return WIFEXITED(status) ? WEXITSTATUS(status) : 128 + WTERMSIG(status);
}

/* 
 * Ajoute les noms passés en paramètres à la liste des machines.
 */

int remote_add(char** param)
{
    param+=2;                               // on saute le «remote» et le «add»
    int i=0;
    int alreadyInList=0;
    while(param[i] != NULL){
      for (int j=0; j<nb_machine; j++) {
	if (strcmp(param[i],tab_machines[j]->name)==0) {
	  alreadyInList=1;
	}
      } 
      if (alreadyInList == 0) {
	//nb_add_machines++;
	remote_machine* rm=malloc(sizeof(remote_machine));
	rm->name=strdup(param[i]);
	if(i+nb_machine>=tab_length){
	    if(tab_length==0)
		tab_length=10;
	    else
		tab_length*=2;
	    tab_machines=realloc(tab_machines,sizeof(remote_machine*)*tab_length);
	}
	tab_machines[nb_machine] = rm;
	nb_machine++;
      }
      alreadyInList=0;
      i++;
    }
    return EXIT_SUCCESS;
}

/*
 *supprime les machines existantes du tableau
 */
int remote_remove(char ** param) 
{
  for (int i=0; tab_machines[i]; i++) 
  {
    free(tab_machines[i]->name);
    free(tab_machines[i]);
  }
  nb_machine=0;
  return EXIT_SUCCESS;
}


/* 
 * Commande affichant la liste des machines connectées 
 */
int remote_list(char ** param)
{
  if (nb_machine == 0)
  {
    fprintf(stderr,"Il n'y a actuellement aucune machine dans la liste des machines distantes connectées.\nUtilisez remote add pour ajouter des machines.\n");
    return EXIT_SUCCESS;
  }
  else 
    for (int i=0; i<nb_machine; i++)
      printf("%s\n", tab_machines[i]->name);
  return EXIT_SUCCESS;
}

/*
 * Fonction exécutant une commande sur le shell d'une machine distante connectée.
 * On donnera en paramètre une machine, et une commande à exécuter avec ses paramètres
 */
int remote_cmd_simple(char** param)
{
    remote_machine* lmachine=NULL;
    param++;//on saute le premier paramèter (toujours remote)
    for(int i=0;i<nb_machine;i++)//on cherche si la machine est enregistrée
    {
	if(tab_machines[i]!=NULL && strcmp(tab_machines[i]->name,*param)==0)
	{
	    lmachine=tab_machines[i];
	    break;
	}
    }
    if (!lmachine) //si non on s'arète
    {
      fprintf(stderr,"aucune machine de ce nom n'est présente dans la liste.\nUtilisez remote list pour obtenir la liste\n");
      return EXIT_FAILURE;
    }
    param++;//on saute le deuxième paramètre (c'est à dire le nom de la machine)
    
    
    char** param_ssh=InitialiserListeArguments();//on va lancer notre shell en mode distant via ssh
    param_ssh=AjouterArg(param_ssh,"ssh");
    param_ssh=AjouterArg(param_ssh,lmachine->name);
    param_ssh=AjouterArg(param_ssh,"./Shell -r");

    char** param_echo=InitialiserListeArguments();//lui passer les paramètres via echo
    param_echo=AjouterArg(param_echo,"echo");
    while(*param!=NULL)
	param_echo=AjouterArg(param_echo,*(param++));
    
    char** param_xcat=InitialiserListeArguments();//et afficher en local le tout avec xcat
    param_xcat=AjouterArg(param_xcat,"./xcat.sh");
    param_xcat=AjouterArg(param_xcat,"-hold");//on ne ferme pas à la fin du programme (sinon on a pas le temps de lire les résultats)
    param_xcat=AjouterArg(param_xcat,"-T");//le titre est le nom de la machine
    param_xcat=AjouterArg(param_xcat,lmachine->name);
    
    Expression* e=ConstruireNoeud(SIMPLE,NULL,NULL,param_ssh);     //on construit les trois commandes
    Expression* echo=ConstruireNoeud(SIMPLE,NULL,NULL,param_echo);
    Expression* xcat=ConstruireNoeud(SIMPLE,NULL,NULL,param_xcat);
    e=ConstruireNoeud(PIPE,echo,e,NULL);//on pipe echo sur ssh
    e=ConstruireNoeud(PIPE,e,xcat,NULL);//et ssh sur xcat
    Contexte* c=malloc(sizeof(Contexte));
    initialiser_contexte(c);
    int ret= get_expr(PIPE)(e,c);//on exécute enfin le tout
    expression_free(e);
    free(c);
    return ret;
}

/* 
 * Fonction exécutant une commande sur le shell de toutes les machines connectées.
 * On donnera en paramètre la commande et ses arguments
 */
int remote_all(char ** param) 
{
    if (nb_machine == 0)
    {
	fprintf(stderr,"Il n'y a actuellement aucune machine dans la liste des machines distantes connectées.\nUtilisez remote add pour ajouter des machines.");
	return EXIT_SUCCESS;
    }
    char * tmp = param[1];
    pid_t* pid=malloc(sizeof(pid_t)*nb_machine);
    for (int i=0; i<nb_machine; i++)                          // pour chaque machine, on exécute cmd_simple avec le nom de la machine et les parametres
    {
	if((pid[i]=fork())==0)
	{
	    param[1] = tab_machines[i]->name;                       // le premier paramètre correspond au nom de la machine  
	    exit(remote_cmd_simple(param));                         // on appelle ensuite cmd_simple avec le nom de la machine et la liste de paramètres.
	}
    }
    param[1] = tmp;
    for (int i=0; i<nb_machine; i++)
	waitpid(pid[i],NULL,0);
    return EXIT_SUCCESS;
}
				  

\end{verbatim}
\label{}
\begin{verbatim}
 
\end{verbatim}


\end{document}

