\documentclass[12pt]{article}

\usepackage[T1]{fontenc} 
\usepackage[utf8]{inputenc}
\usepackage[francais]{babel}
\usepackage{hyperref}
\usepackage{url}
\usepackage{color}
\usepackage[usenames,dvipsnames]{xcolor}
\usepackage{alltt}
\usepackage{tocbibind}


\title{Rapport Projet Programmation Système}
\author{Beziau Paul, Bazalgette Martin, Borde Antoine}

\begin{document}
\maketitle
\tableofcontents

\newpage
\section{Présentation}

L'objectif de ce projet était d'implémenter un Shell capable d'interpréter les expressions de base
utilisable avec bash et d'exécuter des processus. En réalité, une bonne partie du code du Shell
nous était déjà fourni. Tout l'aspect de l'analyse syntaxique était déjà implémenté dans les fichiers
donnés dans le sujet.\newline

On peut diviser le travail à effectuer en 4 parties qui sont:
\begin{itemize}
 \item Implémenter la possibilité pour notre Shell d'exécuter des
 commandes externes ou des programmes en tout genre n'appartenant pas au Shell
 \item Donner au Shell la capacité d'interpréter les expressions courantes
 utilisables sous bash, comme les redirections ou les pipes
 \item Écrire un certain nombre de commandes interne (des fonctions appartenant
 au Shell en lui-même) tout en faisant attention à les rendre utilisables avec
 les expressions précédemment citées.
 \item L'implémentation d'une commande interne "remote" qui nous permettra d'exécuter
 et contrôler des shells distants en utilisant ssh.\newline 
\end{itemize}

Nous verrons au travers de ce rapport comment nous avons choisi d'implémenter ces différentes
fonctionnalités.

\newpage

\section{Commande externe}
 
 Le début de ce projet avait déjà été bien entamé en tp. Nous nous sommes donc
 inspirés de ce qui avait été fait en classe. Nous avons créé une structure "contexte"
 qui contiendra à terme tous les éléments dont nous aurons besoin. \ref{Eval.h}
 
 Ensuite, nous avons implémenté une deuxième structure "assoc" qui possède un champ "expr\_t"
 et un pointeur de fonction reliant le type d'expression a la fonction de traitement
 qui lui est associé.
 
 Pour finir, nous avons créé la fonction "get\_expr" qui retourne la fonction associée au 
 type de l'expression et l'appel en lui passant les paramètres de l'expression et son 
 contexte.\ref{struct et fonctionnement}
 
 Ce travail préparatoire permettra de rendre beaucoup plus aisée la suite du développement
 du projet
 
 En effet ne nous restait alors plus qu'à implémenter une fonction pour intégrer le comportement
 de chaque expression.
 
 \subsection{Commande simple en avant-plan}
 
 L'exécution de commande simple en avant plan est implémentée avec un simple fork().
 On récupère les arguments de la commande rentrée et l'on fait un "execvp" de cette commande
 avec ces paramètres du côté de fils. Pendant ce temps le père attend la fin de l'exécution
 et récupère la valeur de retour de son fils.\ref{expr_simple}
 
 \subsection{Commande simple en arrière plan}
 
 Pour traiter des commandes en arrière-plan, nous avons rajouté un champ "background"
à notre structure contexte qui passe à vraie si le caractère \& est interprété.
 Ensuite, il suffit de dire que dans ce contexte on lance le fork() mais on n’attend pas
la fin de l'exécution du fils.
 
 
 \subsection{Élimination des zombies}

 ...............paul
 
 \ref{main}
 

\newpage
\section{Expression}

Dans cette partie, le travail consiste à implémenter l'interprétation d'expression courante
normalement utilisable avec bash tel que les redirections ">" ou les pipes "|".
En réalité, ces expressions sont déjà reconnues par le Shell. Il ne nous reste plus qu'à utiliser
nos structures pour relier chaque expression à son comportement spécifique et d'implémenter le
dit comportement.

Comme expliquer plus haut le gros travail préparatoire que nous avons effectué nous permettra
de traiter avec une relative simplicité cette partie. En effet, il nous suffira la plupart du temps 
d'exploiter l'arborescence déjà implémentée dans le sujet, ainsi que nos structures, pour concevoir
les différents comportements.\ref{autre expression}

\subsection{Les séquences}

Une expression séquentielle consiste à exécuter la commande de gauche puis celle de droite.
Il nous suffit donc d'appeler dans l'ordre avec notre fonction "get\_expr" le fils gauche puis le fils
droit de notre expression ";".

\subsection{les expressions booleene}

Nous avons suivi le même principe pour les expressions booléennes. On appelle le fils gauche puis
le fils droit de l'expression.
À la différence que le fils droit n'est appelé que si la commande gauche à bien était exécuté dans le
cas du "et".Et le fils droit n'est appelé qu’en cas d'échec du fils gauche dans le cas du "ou"

\subsection{Les sous-expressions}

Il s'agit sans doute de l'expression la plus simple à gérer. Il suffit d'appeler le fils
gauche de l'expression. La gestion de l'arborescence fera le reste.

\subsection{Les redirections}

Pour gérer les redirections, nous avons rajouté dans notre structure contexte des descripteurs
de fichier correspondant aux entrées et sortie standard des processus. Il nous suffit ensuite 
d'ouvrir le fichier passé en paramètre sur les bons descripteurs et d'appeler la commande.
Les entrées et sorties seront ensuite remises à zéro par notre fonction "swapfd".

\subsection{Les pipes}

........paul


\subsection{Les expressions récursives}

............paul

\newpage
\section{Commande interne}

Dans cette partie, nous avons dû implémenter directement le comportement de quelques
commandes dans notre Shell. Ces commandes n'étaient pas très difficiles à implémenter. La plupart
du temps, il suffisait de trouver le bon appel système à utiliser. À noter tout de même
que nous n'avons implémenté que les fonctionnalités de base offertes par ces commandes, nous
ne nous sommes pas amusés à ajouter la gestion de tous les paramètres, ce qui aurait 
représenté un travail très fastidieux.\ref{cmd_de_base}

\subsection{Fonction}
\subsubsection{echo}
 La commande "echo" n'est rien d'autre qu'une fonction qui écrit les paramètres qu'elle reçoit
 sur la sortie standard. Un simple printf suffit pour cette commande.
 
\subsubsection{date}
 Pour cette commande, nous avons dû faire quelques recherches sur les fonctions fournies par la librairie "time".
 Nous avons décidé de créer une structure "tm" et d'utiliser la fonction "strftime" qui
 sert justement à afficher l'heure selon son propre format qu'on peut définir à la volée.
 
 \subsubsection{cd}
 La commande "cd" a requis d'utiliser l'appel système "chdir" qui ouvre le répertoire passé en paramètre
 Nous avons veillé à récupérer la variable d'environnement "HOME" avec getenv pour les cas ou "cd" est
 appelé sans paramètre. Nous avons également tenté de prévenir d'éventuelles erreurs(comme les
 répertoires non existants).
 
 \subsubsection{pwd}
 Cette commande appelle la fonction getcwd qui copie le chemin absolu du répertoire 
 courant dans un buffer.
 Le buffer est initialisé avec la taille PATH\_MAX, qui est une macro du fichier limites.h,
 et qui correspond à la taille maximale d'un chemin. Il suffit ensuite d'afficher le buffer.
 
 \subsubsection{hostname}
 Cette fonction a été implémentée d'une manière semblable à la fonction PWD.
Tout d'abord, nous avons du récupérer le nom de l'hôte local grâce à l'appel système gethostname, qui copie ce nom dans un buffer. 
La taille du buffer est initialisée à l'aide de la macro HOST\_NAME\_MAX qui correspond à la taille maximale d'un nom de localhost. 
On affiche ensuite ce dernier.
 
 \subsubsection{exit}
 La commande exit permet de quitter le Shell. 
 Elle peut être exécutée avec ou sans paramètres. 
 S'il n'y a pas de paramètre, on ferme le Shell en renvoyant la valeur 0. 
 Sinon, on ferme le Shell en renvoyant la valeur entière du premier paramètre. Il est donc inutile de donner plusieurs paramètres à cette commande.
 
 \subsubsection{kill}
 Pour la commande kill nous avons simplement utilisé l'appel système kill. 
 Notre commande peut gérer le paramètre de choix de signal à envoyer. Il peut l'envoyer à tous les processus
 rentrés en paramètres.
 Pour gérer l'erreur en cas d'absence de paramètres nous avons touché directement à la valeur
 de errno la passant à EINVAL (invalid argument).
 
 \subsubsection{history}
 La commande history renvoie l'historique des commandes entrées par l'utilisateur. 
 Elle nécessite la gestion d'un historique dans le Shell. 
 Pour cela il a fallu inclure le fichier history.h permettant d'utiliser un historique. 
 Pour ce faire, on appelle dans un premier temps la fonction using\_history dans le main avant la boucle principale.
 Ainsi on pourra utiliser les fonctions de gestions d'historique. 
 À chaque fois qu'une fonction est exécutée, elle est ajoutée à l'historique grâce à la fonction add\_history. 
 Ensuite, notre fonction history récupère la liste des fonctions appelées, et les affiches une par une. 
 On peut également exécuter la commande avec un argument n, pour afficher uniquement les n dernières commandes.
 
 \subsection{Intégration au Shell}
 
 La difficulté de cette partie n'était pas tant l'implémentation des fonctions en elles mêmes,
 mais surtout le fait de rendre compatible tout ce qui avait été fait avant avec ces commandes.
 En effet les commandes internes n'utilisent pas de fork il faut donc gérer leurs comportements
 de manière particulière.
 ......................paul
 
 \ref{expr_simple}
 

\newpage
\section{Remote Shell}

..........paul


\ref{remote shell}





\newpage
\section{Conclusion}

Notre plus grande réussite dans ce projet est trés clairement le travail éffectué au
début du projet. En effet nous avons sus tout de suite préparer un terrain propice au
dévellopement de notre shell grace a l'implémentation de notre systeme de structure et
de fonction définnisant le comportement des expressions\ref{struct et fonctionnement}.
La preuve en est qu'arriver a l'étape de la gestion de la récursivité des expressions, 
cette option étais en fait deja fonctionnel.




\newpage
\section{Annexes}

Ci-joint en annexes le code source du projet. Nos fichiers sont disponible sur le dépot git suivant:
 \href{https://github.com/beziarum/shell.git}{ici}.

 
\subsection{Shell.c}
\label{main}
\begin{alltt}
 
\end{alltt}


\subsection{Evaluation.c}

\subsubsection{Structure et fonctionnement}
\label{struct et fonctionnement}
\begin{alltt}
 
\end{alltt}

\subsubsection{Expression simple}
\label{expr_simple}
\begin{alltt}
 
\end{alltt}

\subsubsection{Autre Expressions}
\label{autre expression}
\begin{alltt}
 
\end{alltt}

\subsection{Evaluation.h}
\label{Eval.h}
\begin{alltt}
 
\end{alltt}

\subsection{Commandes-internes.c}

\subsubsection{Les commandes internes de bases}
\label{cmd_de_base}
\begin{alltt}
 
\end{alltt}

\subsubsection{remote shell}
\label{remote shell}
\begin{alltt}
 
\end{alltt}

\subsection{Commandes-internes.h}
\label{cmd_interne.h}
\begin{alltt}
 
\end{alltt}



\end{document}


