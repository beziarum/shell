\documentclass[12pt]{article}

\usepackage[T1]{fontenc} 
\usepackage[utf8]{inputenc}
\usepackage[francais]{babel}
\usepackage{graphicx}
\usepackage{hyperref}
\usepackage{url}
\usepackage{color}
\usepackage[usenames,dvipsnames]{xcolor}
\usepackage{alltt}
\usepackage{tocbibind}


\title{Rapport Projet Programmation Système}
\author{Beziau Paul, Bazalgette Martin, Borde Antoine}

\begin{document}
\maketitle
\tableofcontents

\newpage
\section{Présentation}

L'objectif de ce projet étais d'implémenter un shell capable d'interpréter les expression de base
utilisable avec bash et d'éxétcuter des procéssus. En réalité une bonne partie du code du shell
nous été deja fournit. Tout l'aspect de l'analyse syntaxique été déja implémenté dans les fichier
donné dans le sujet.\newline

On peut diviser le travail à effectuer en 4 partie qui sont:
\begin{itemize}
 \item Implémenter la possibilité pour notre shell d'éxécuter des
 commande externe ou des progrmme en tout genre n'appartenant pas au shell
 \item Donner au shell la capacité d'interpréter les expression courante
 utilisable sous bash, comme les redirections ou les pipes
 \item Ecrire un certain nombre de commande interne (des fonctions appartennant
 au shell en lui meme) tout en faisant attention à les rendre utilisable avec
 les expressions précédement cité
 \item L'implémentation d'une commande interne "remote" qui nous permetra d'éxécuter
 et controler des shells distants en utilisant ssh.\newline 
\end{itemize}

Nous verrons au travers de ce rapport comment nous avons choisit d'implémenter ces différentes
fonctionnalité.

\newpage
\section{Commande externe}


\newpage
\section{Expression}


\newpage
\section{Commande interne}

Dans cette partie nous avons du implémenter directement le comportement de quelque
commande dans notre shell. Ces commandes n'etais pas trés dur a implémenter. La plupart
du temps il suffisait de trouver le bon appel systeme à utilisé. A noter tout de meme
que nous n'avons implémenté que les fonctionnalités de base offerte par ces commandes, nous
ne nous somme pas amusé a ajouter la gestion de tout les parametres ce qui aurait 
représenté un travail trés fastidieux.

\subsection{Fonction}
\subsubsection{echo}
 La commande echo n'est rien d'autre qu'une fonction qui écrit les parametres quelle reçoit
 sur la sortie standard. Un simple printf suffit pour cette commande.
 
\subsubsection{date}
 Pour cette commande nous avons du faire quelque recherche sur les fonction fournit par time.h
 Nous avons décider de créer une structure tm et d'utiliser la fonction strftime qui
 sert justement a afficher l'heure celon son propre formt qu'on peut définir a la volé.
 
 \subsubsection{cd}
 La commande cd a requis d'utiliser l'appel systeme chdir qui ouvre le repertoir passé en paramétre
 Nous avons veiller a récupéré la variable d'environement HOME avec getenv pour les cas ou cd est
 appellé sans parametres. Nous avons également tenté de prévenir d'éventuels erreurs(comme les
 repertoir non éxistant).
 
 \subsubsection{pwd}
 Cette commande a juste besoin de recupéré le répertoire courant nous avons donc utilisé
 la fonction getcwd qui retourne ce dernier. La taille maximal du buffer acceuillent le chemin
 du repertoire est....
 
 \subsubsection{hostname}
 
 \subsubsection{exit}
 
 \subsubsection{kill}
 Pour la commande kill nous avons simplement utilisé l'apelle systeme kill. Notre commande
 peut gérer le parametres de choix de signal a envoyer. Il peut l'evoyer a tout les processus
 rentré en parametres.
 Pour gérer l'erreur en cas d'absence de parametres nous avons toucher directement a la valeur
 de errno la passant a EINVAL invalid argument.
 
 \subsubsection{history}
 
 \subsection{Intégration au shell}
 
 La difficulté de cette partie n'etais pas tant l'implémentation des fonctions en elle meme
 surtout le fait de rendre compatible tous se qui avait été fait avant avec ces commandes.
 En effet les commande interne n'utilise pas de fork il faut donc gérer leur comportement
 de maniére particuliére.

\newpage
\section{Remote shell}


\newpage
\section{Conclusion}



\end{document}
